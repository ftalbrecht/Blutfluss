\documentclass[a4paper,12pt]{scrartcl}

\usepackage{hyperref} % Links in pdf
\usepackage[utf8]{inputenc}
\usepackage[ngerman]{babel}
\usepackage{amsmath}
\usepackage{amssymb}
\usepackage{graphicx}
\usepackage[locale=DE]{siunitx} % Einheiten mit \SI{5,2\pm0,2}{cm}
\usepackage{icomma} % 2,5 statt 2, 5
\usepackage[ngerman]{cleveref}
\sisetup{separate-uncertainty} % \pm sieht gut aus
\graphicspath{{graf/}} % Standardverzeichnis für Grafiken

\renewcommand{\labelenumi}{\arabic{enumi})}
\renewcommand{\labelenumii}{\alph{enumii})}

\begin{document}
\section*{Realistische Werte}
\subsection*{Eigenschaften von Blut}
\begin{itemize}
  \item Dichte $\rho = \SI{1060}{\frac{kg}{m^3}}$\\
  (Quelle: \url{https://en.wikipedia.org/wiki/Blood})

  \item Viskosität $\mu = \SI{3E-3}{} - \SI{4e-3}{Pa s}$\\
  (Quelle: \url{https://en.wikipedia.org/wiki/Hemorheology})

  \item Poisson Ratio $\nu = \frac{1}{2}$\\
  (Quelle: Buch Cardiovascular Mathematics)

\end{itemize}
\subsection*{Eigenschaften von Adern}
\begin{itemize}
  \item Young's modulus / Elastizitätsmodul $E \approx \SI{150}{kPa}$\\
  (Quelle: \url{http://www.sciencedirect.com/science/article/pii/S2211568413000338}, Fig. 8)
  \item
  The larger arteries ($>$10 mm diameter) are generally elastic and the smaller ones (0.1–10 mm) tend to be muscular.\\
  (Quelle: \url{https://en.wikipedia.org/wiki/Artery})
  \item Radius $r_0$ zwischen 10 und 0.1 mm (s.o.). Für Terminal Vessels (Kapilare) auch bis zu 0.01mm\\
  (Quelle: \url{http://systemdesign.ch/wiki/Blutkreislauf})
  \item Flussgeschwindigkeit in Terminal Vessels: 0.3mm/s\\
  (Quelle: \url{http://systemdesign.ch/wiki/Blutkreislauf})
  \item typischer Blutdruck: \SI{13}{kPa}\\
  (Quelle: \url{https://de.wikipedia.org/wiki/Blutdruck})
  \item Wandstärke $h_0 \approx r_0 / 10$ (Schätzung)
\end{itemize}

\end{document}
