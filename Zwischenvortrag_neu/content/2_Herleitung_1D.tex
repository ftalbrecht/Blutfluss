\section{Herleitung des 1D-Modells}




\begin{frame}\frametitle{Herleitung}

  Reynolds Transport Theorem:
  \begin{align}
    \frac{d}{dt} \int_{V_t} f dV = \int_{V_t} \frac{\partial f}{\partial t}dV + \int_{\partial V_t} f \vec u_w \cdot \vec n d\sigma \label{10.1} % 10.1
  \end{align}

  \begin{center}
      \includegraphics[width=0.5\textwidth]{vessel}
  \end{center}

\end{frame}


\begin{frame}\frametitle{Herleitung}

  Mittelung über Querschnitt:
  \begin{align}
    \bar f = \frac{1}{A} \int_S f d\sigma % 10.3
  \end{align}
  Damit lhs:
  \begin{align}
    \frac{d}{dt} \int_{V_t} f dV = \int_{x_1}^{x_2} \frac{\partial}{\partial t} (A \bar f) dx % 10.5
  \end{align}

\end{frame}


\begin{frame}\frametitle{Herleitung}
  aus rhs mit Gauß-Theorem:
  \begin{equation}
  \begin{aligned}
    \int_{\partial V_t} f \vec u_w \cdot \vec n d\sigma = \int_{\partial V_{t, w}} f \vec w \cdot \vec n d\sigma &- \int_{x_1}^{x_2} \frac{\partial}{\partial x} \left[  A (\bar{f u_1}) \right] dx \\&+ \int_{V_t} \nabla \cdot (f \vec u) dV
  \end{aligned} % 10.6
\end{equation} %TODO: ist die Formel zu viel?
In Gl. (\ref{10.1}) \& x-Integration wegfallen lassen:
\begin{align}
\boxed{
\frac{\partial}{\partial t} (A \bar f) + \frac{\partial}{\partial x} \left[ A (\bar{f u_1}) \right] = \int_S \left[ \frac{\partial f}{\partial t}
+ \nabla \cdot (f\vec u) \right] d \sigma
+ \int_{\partial S} f \vec w \cdot \vec n d \gamma} % 10.7
\end{align}
\end{frame}

\begin{frame}\frametitle{Herleitung}
Massenerhaltung:\\~\\
Setze $f=1$ und $\nabla \cdot \vec u = 0$ (inkompressibel):
\begin{align}
  \frac{\partial A}{\partial t} + \frac{\partial}{\partial x} (A \bar u_1) = \int_{\partial S} \vec w \cdot \vec d d\gamma % 10.8
\end{align}
\end{frame}

\begin{frame}\frametitle{Herleitung}
  Impulserhaltung:\\~\\
  Setze $f=u_1$, $\nabla \cdot \vec u = 0$, $\rho=$const. und Annahme, dass Viskousitätskraft linear in $\bar{u}$ ist:
\begin{align}
\boxed{
\frac{\partial}{\partial t} (A \bar u_1) + \frac{\partial}{\partial x} \left[ A \alpha \bar{u}_1^2 \right] = A \bar{f}^b_1 - \frac{A}{\rho}\left(\frac{\partial \vec{p}}{\partial x}\right)- K_R \bar{u}_1 + \int_{\partial S}u_1 \vec{w}\cdot \vec{n}\text{d}\sigma}
\end{align}
  mit der Materialableitung $\frac{D}{Dt}=\frac{\partial}{\partial t}+\vec{u}\cdot \nabla$ und dem Korrekturfaktor $\bar{u^2}_1=\alpha \cdot \bar{u}_1^2$. %TODO sagen: alpha?1 z.b. flaches profil
  \end{frame}
%  \begin{frame}\frametitle{Herleitung}
%   Mit dem divergenz Theorem, der "constitutive" Gleichung, der Körperkraft pro Volumeneinheit $f^b$ und wieder weglassen von $x$-Integration:
%  \begin{align}
%  \int_S\frac{Du_1}{Dt}\text{d}\sigma = \int_S\left[f_1^b+\frac{1}{\rho}\left(-\frac{\partial p}{\partial x}+d_1\right)\right]
%  \end{align}
%  wobei $\nabla \cdot \vec{D}=\vec{d}$, mit $\vec{D}$ der deviatorische Stresstensor ist und $d_1$ die x-Komp. darstellt.
%\end{frame}
%\begin{frame}
%\frametitle{Herleitung}
%Da i.A. $\bar{u}_1^2\neq \bar{u^2}_1$, aber nur $\bar{u}_1^2$ bekannt, Korrekturterm:
%\begin{align}
%\bar{u^2}_1=\alpha \cdot \bar{u}_1^2
%\end{align}
%Die Viskousitätkraft $d_1$ ist lin. von $\bar{u}_1$ $\Rightarrow$
%\begin{align}
%\boxed{
%\frac{\partial}{\partial t} (A \bar u_1) + \frac{\partial}{\partial x} \left[ A \alpha \bar{u}_1^2 \right] = A \bar{f}^b_1 - \frac{A}{\rho}\left(\frac{\partial \vec{p}}{\partial x}\right)- K_R \bar{u}_1 + \int_{\partial S}u_1 \vec{w}\cdot \vec{n}\text{d}\sigma}
%\end{align}
%\end{frame}
\begin{frame}
\frametitle{Wall-Mechaniken}
Annahme: statisches Ggw. in radialen Richtung im Zylinder:
\begin{align}
p=P_\text{ext}+\beta(\sqrt{A}-\sqrt{A_0})
\end{align}
mit $\beta = \frac{\sqrt{\pi}h_0E}{(1-\nu^2)A_0}$.
\end{frame}
\begin{frame}
\frametitle{\textbf{Endformel}}
Mit $\alpha=1$ folgt:
\begin{align}
\boxed{\frac{\partial \vec{U}}{\partial t}+\frac{\partial \vec{F}}{\partial x}=\vec{S}(\vec{U})}
\end{align}
mit :
\begin{align*}
\vec{U}=\begin{pmatrix} A \\ u \end{pmatrix} \hspace{1cm} \vec{F}=\begin{pmatrix} Au \\ \frac{u^2}{2} + \frac{p}{\rho}
\end{pmatrix} \hspace{1cm} \vec{S}=\begin{pmatrix} 0 \\ -K_R \frac{u}{A}
\end{pmatrix}
\end{align*}
\end{frame}