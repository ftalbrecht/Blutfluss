\section{Das 0D-Modell}

\begin{frame}
\frametitle{0D-Modell}
\begin{equation}
  \begin{aligned}
    C \frac{dP_1}{dt} &= Q_1 - Q_2\\
    L \frac{dQ_2}{dt} &= P_1 - P_2 - RQ_2
  \end{aligned}
\end{equation}
\pause
Analog zu elektrischen Bauteilen:
\begin{center}
  \includegraphics[width=0.6\textwidth]{kirchhoff.png}
\end{center}
\end{frame}

\begin{frame}
\frametitle{0D-Modell}
  \begin{center}
  	\begin{tabular}[!htb]{c c c}
  		Hydraulisch	&	Variable	&	Elektrisch\\
  		\hline
  		Druck	&	$P$	&	Spannung\\
  		Flussrate	&	$Q$	&	Stromstärke\\
  		Blut-Viskosität	&	$R$	&	Widerstand\\
  		Blut-Trägheit	&	$L$	&	Induktivität\\
  		Ader-Elastizität	&	$C$	&	Kapazität
  	\end{tabular}
  \end{center}

\end{frame}
