\section{Herleitung des 1D-Modells}

\begin{frame}
\frametitle{Notizen}

HERLEITUNG DES 1D Models

Reynolds Transport Theorem Gl. 10.1
  undurchlässige Wand

Flächenmittelung Gl. 10.3

Welche Annahmen getroffen?

Gl. 10.5

Gl. 10.7 und die Gleichung davor... Irgendwie, damit der Übergang von
10.1 zu 10.7 klar wird

Conservation of mass
Conservation of momentum
  p einführen über Stress Tensor (Annahme)

gemitteltes $u^2$ annähern (Annahme)
gemitteltes d annähern (10.19) (Annahme)

3 Unbekannte in 2 Gleichungen -> Eine zusätzliche Bedingung -> Simplified
Models of wall mechanics, einfache Beziehung angeben (eine Zeile) (Annahme)



\end{frame}


\begin{frame}\frametitle{Herleitung}

  Reynolds Transport Theorem:
  \begin{align}
    \frac{d}{dt} \int_{V_t} f dV = \int_{V_t} \frac{\partial f}{\partial t}dV + \int_{\partial V_t} f \vec u_w \cdot \vec n d\sigma \label{10.1} % 10.1
  \end{align}

  \begin{center}
      \includegraphics[width=0.5\textwidth]{vessel}
  \end{center}

\end{frame}


\begin{frame}\frametitle{Herleitung}

  Mittelung über Querschnitt:
  \begin{align}
    \bar f = \frac{1}{A} \int_S f d\sigma % 10.3
  \end{align}
  Damit lhs:
  \begin{align}
    \frac{d}{dt} \int_{V_t} f dV = \int_{x_1}^{x_2} \frac{\partial}{\partial t} (A \bar f) dx % 10.5
  \end{align}

\end{frame}


\begin{frame}\frametitle{Herleitung}
  aus rhs mit Gauß-Theorem:
  \begin{equation}
  \begin{aligned}
    \int_{\partial V_t} f \vec u_w \cdot \vec n d\sigma = \int_{\partial V_{t, w}} f \vec w \cdot \vec n d\sigma &- \int_{x_1}^{x_2} \frac{\partial}{\partial x} \left[  A (\bar{f u_1}) \right] dx \\&+ \int_{V_t} \nabla \cdot (f \vec u) dV
  \end{aligned} % 10.6
\end{equation} %TODO: ist die Formel zu viel?
In Gl. (\ref{10.1}) \& x-Integration wegfallen lassen:
\begin{align}
\boxed{
\frac{\partial}{\partial t} (A \bar f) + \frac{\partial}{\partial x} \left[ A (\bar{f u_1}) \right] = \int_S \left[ \frac{\partial f}{\partial t}
+ \nabla \cdot (f\vec u) \right] d \sigma
+ \int_{\partial S} f \vec w \cdot \vec n d \gamma} % 10.7
\end{align}
\end{frame}

\begin{frame}\frametitle{Herleitung}
Massenerhaltung:\\~\\
Setze $f=1$ und $\nabla \cdot \vec u = 0$ (inkompressibel):
\begin{align}
  \frac{\partial A}{\partial t} + \frac{\partial}{\partial x} (A \bar u_1) = \int_{\partial S} \vec w \cdot \vec d d\gamma % 10.8
\end{align}
\end{frame}

\begin{frame}\frametitle{Herleitung}
  Impulserhaltung:\\~\\
  Setze $f=u_1$ und $\nabla \cdot \vec u = 0$:
\end{frame}
