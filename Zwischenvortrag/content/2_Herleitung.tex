\section{Herleitung des 0D-Modells}

\begin{frame}
\frametitle{Herleitung}

HERLEITUNG DES 1D Models

Reynolds Transport Theorem Gl. 10.1
  undurchlässige Wand

Flächenmittelung Gl. 10.3

Welche Annahmen getroffen?

Gl. 10.5

Gl. 10.7 und die Gleichung davor... Irgendwie, damit der Übergang von
10.1 zu 10.7 klar wird

Conservation of mass
Conservation of momentum
  p einführen über Stress Tensor (Annahme)

gemitteltes u^2 annähern (Annahme)
gemitteltes d annähern (10.19) (Annahme)

3 Unbekannte in 2 Gleichungen -> Eine zusätzliche Bedingung -> Simplified
Models of wall mechanics, einfache Beziehung angeben (eine Zeile) (Annahme)



\end{frame}



\begin{frame}\frametitle{Herleitung}

  HERLEITUNG DES 0D Models

  Integral für Ortsmittelung angeben, Größen mit Dach einführen

  S. 375 werden zwei Annahmen gemacht, die drauf bringen

  HIER simple Wall Mechanics rein bringen

\end{frame}
